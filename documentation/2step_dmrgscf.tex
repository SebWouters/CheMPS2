\documentclass[11pt,a4paper]{article}
\usepackage[utf8]{inputenc}
\usepackage{amsmath}
\usepackage{amsfonts}
\usepackage{amssymb}
\usepackage{fullpage}
\usepackage{hyperref}
\usepackage{braket}
\author{Sebastian Wouters}
\title{Orbital gradient and Hessian for DMRG-SCF in \textsc{CheMPS2}}
\date{January 22, 2015}
\begin{document}
\maketitle

The calculation of the orbital gradient and orbital-orbital Hessian for DMRG-SCF is based on P.E.M. Siegbahn, J. Alml\"of, A. Heiberg and B.O. Roos, \textit{J. Chem. Phys.} \textbf{74}, 2384 (1981); \url{http://dx.doi.org/10.1063/1.441359}. The basic idea is to express the energy with the unitary group generators:
\begin{eqnarray}
\hat{E}_{pq} & = & \sum\limits_{\sigma} \hat{a}^{\dagger}_{p \sigma} \hat{a}_{q \sigma} \\
\left[ \hat{E}_{pq} , \hat{E}_{rs} \right] & = & \delta_{qr} \hat{E}_{ps} - \delta_{ps} \hat{E}_{rq} \\
\hat{E}^{-}_{pq} & = & \hat{E}_{pq} - \hat{E}_{qp} \\
\hat{T} & = & \sum\limits_{p<q} x_{pq} \hat{E}^{-}_{pq} \\
E(x) & = & \braket{0 \mid e^{\hat{T}} \hat{H} e^{-\hat{T}} \mid 0 } \\
\left. \frac{\partial E(x)}{\partial x_{ij}} \right|_{0} & = & \braket{ 0 \mid \left[ \hat{E}_{ij}^{-}, \hat{H} \right] \mid 0 } \\
\left. \frac{\partial^2 E(x)}{\partial x_{ij} \partial x_{kl}} \right|_{0} & = & \frac{1}{2} \braket{ 0 \mid  \left[ \hat{E}_{ij}^{-}, \left[ \hat{E}_{kl}^{-}, \hat{H} \right] \right] \mid 0 } + \frac{1}{2} \braket{ 0 \mid  \left[ \hat{E}_{kl}^{-}, \left[ \hat{E}_{ij}^{-}, \hat{H} \right] \right] \mid 0 }
\end{eqnarray}
The variables $x_{pq}$ only connect orbitals with the same irrep ($I_p=I_q$). Assuming that DMRG is exact, $x_{pq}$ in addition only connects orbitals when they belong to different occupation blocks: \{occupied, active, virtual\}. With some algebra, the derivatives can be rewritten. Real-valued symmetric one-electron integrals $h_{ij}$ and real-valued eightfold permutation symmetric two-electron integrals $(ij \mid kl)$ are assumed (chemical notation for the latter).
\begin{eqnarray}
\Gamma^{2A}_{ijkl} & = & \sum\limits_{\sigma \tau} \braket{ 0 \mid \hat{a}^{\dagger}_{i \sigma} \hat{a}_{j \tau}^{\dagger} \hat{a}_{l \tau} \hat{a}_{k \sigma} \mid 0} \\
\Gamma^1_{ij} & = & \sum\limits_{\sigma} \braket{ 0 \mid \hat{a}^{\dagger}_{i \sigma} \hat{a}_{j \sigma} \mid 0} \\
\left. \frac{\partial E(x)}{\partial x_{ij}} \right|_{0} & = & 2 \left( F_{ij} - F_{ji} \right) \\
F_{pq} & = & \sum\limits_{r} \Gamma_{pr}^{1} h_{qr} + \sum\limits_{rst} \Gamma^{2A}_{psrt} (qr \mid st) \\
\left. \frac{\partial^2 E(x)}{\partial x_{ij} \partial x_{kl}} \right|_{0} & = & w_{ijkl} - w_{jikl} - w_{ijlk} + w_{jilk} \\
w_{pqrs} & = & \delta_{qr} \left( F_{ps} + F_{sp} \right) + \widetilde{w}_{pqrs} \\
\widetilde{w}_{pqrs} & = & 2 \Gamma^1_{pr} h_{qs} + 2 \sum\limits_{\alpha \beta} \left( \Gamma^{2A}_{r \alpha p \beta} (qs \mid \alpha \beta ) + \left( \Gamma^{2A}_{r \alpha \beta p} + \Gamma^{2A}_{r p \beta \alpha} \right) (q \alpha \mid s \beta ) \right)
\end{eqnarray}
In the calculation of $F_{pq}$, the indices $prst$ can only be occupied or active due to their appearance in the density matrices, and the only index which can be virtual is hence $q$. Moreover, due to the irrep symmetry of the integrals and density matrices, $F_{pq}$ is diagonal in the irreps: $I_p = I_q$. Alternatively, this can be understood by the fact that $x_{pq}$ only connects orbitals with the same irrep.

In the calculation of $\widetilde{w}_{pqrs}$, the indices $pr\alpha\beta$ can only be occupied or active due to their appearance in the density matrices, and the only indices which can be virtual are hence $qs$. Together with the remark for $F_{pq}$, this can save time for the two-electron integral rotation. Moreover, as $x_{pq}$ only connects orbitals with the same irrep, $I_p = I_q$ and $I_r = I_s$ in $\widetilde{w}_{pqrs}$.

By rewriting the density matrices, the calculation of $F_{pq}$ and $\widetilde{w}_{pqrs}$ can be simplified. In the following, occ and act denote the doubly occupied and active orbital spaces, respectively.
\begin{eqnarray}
\Gamma^1_{ij} & = & 2 \delta_{ij}^{\text{occ}} + \Gamma^{1,\text{act}}_{ij} \\
\Gamma^{2A}_{ijkl} & = & \Gamma^{2A,\text{act}}_{ijkl} + 2 \delta_{ik}^{\text{occ}} \Gamma^{1,\text{act}}_{jl} - \delta_{il}^{\text{occ}} \Gamma^{1,\text{act}}_{jk} \nonumber \\
& + & 2 \delta_{jl}^{\text{occ}} \Gamma^{1,\text{act}}_{ik} - \delta_{jk}^{\text{occ}} \Gamma^{1,\text{act}}_{il} + 4 \delta_{ik}^{\text{occ}} \delta_{jl}^{\text{occ}} - 2 \delta_{il}^{\text{occ}} \delta_{jk}^{\text{occ}}
\end{eqnarray}
Define the following symmetric Coulomb and exchange matrices:
\begin{eqnarray}
Q^{\text{occ}}_{ij} & = & \sum\limits_{s \in \text{occ}} \left[ 2 (ij \mid ss) - (is \mid js ) \right] \\
Q^{\text{act}}_{ij} & = & \sum\limits_{st \in \text{act}} \frac{1}{2} \Gamma^{1,\text{act}}_{st} \left[ 2 (ij \mid st) - (is \mid jt ) \right] 
\end{eqnarray}
They can be calculated efficiently by (1) rotating the occupied and active density matrices from the current basis to the original basis, (2) contracting the rotated density matrices with the two-electron integrals in the original basis, and (3) rotating these contractions to the current basis. The constant part and the one-electron integrals of the active space Hamiltonian are:
\begin{eqnarray}
 \widetilde{E}_{\text{const}} & = & E_{\text{const}} + \sum\limits_{s \in occ} \left( 2 h_{ss} + Q_{ss}^{\text{occ}} \right) \\
 \widetilde{h}_{ij} & = & h_{ij} + Q_{ij}^{\text{occ}} 
\end{eqnarray}
The calculation of $F_{pq}$ simplifies significantly:
\begin{eqnarray}
p \in \text{occ} & : & F_{pq} = 2 \left( h_{qp} + Q^{\text{occ}}_{qp} + Q^{\text{act}}_{qp} \right) \\
p \in \text{act} & : & F_{pq} = \sum\limits_{r \in \text{act}} \Gamma^{1,\text{act}}_{pr} \left[ h_{qr} + Q^{\text{occ}}_{qr} \right] +  \sum\limits_{rst \in \text{act}} \Gamma^{2A,\text{act}}_{psrt} (qr \mid st)
\end{eqnarray}
And the calculation of $\widetilde{w}_{pqrs}$ as well (remember that $I_p = I_q$ and $I_r = I_s$):
\begin{eqnarray}
(p,r) \in (\text{occ,occ}) & : & \widetilde{w}_{pqrs} = 4 \delta_{pr}^{\text{occ}} \left[ h_{qs} + Q^{\text{occ}}_{qs} + Q^{\text{act}}_{qs} + 3 (qp \mid sp) - (qs \mid pp) \right] \nonumber \\
      & + & 4 ( 1 - \delta_{pr}^{\text{occ}} ) \left[ 4 (qp \mid sr) - ( qs \mid pr ) - ( qr \mid sp ) \right] \\
(p,r) \in (\text{act,act}) & : & \widetilde{w}_{pqrs} = 2 \Gamma^{1,\text{act}}_{rp} \left[ h_{qs} + Q^{\text{occ}}_{qs} \right] \nonumber \\
      & + & 2 \sum\limits_{\alpha\beta \in \text{act}} \left[ \left( \Gamma^{2A,\text{act}}_{r \alpha p \beta} (qs \mid \alpha \beta ) + \left( \Gamma^{2A,\text{act}}_{r \alpha \beta p} + \Gamma^{2A,\text{act}}_{r p \beta \alpha} \right) (q \alpha \mid s \beta ) \right) \right] \\
(p,r) \in (\text{act,occ}) & : & \widetilde{w}_{pqrs} = 2 \sum\limits_{\alpha \in \text{act}} \Gamma^{1,\text{act}}_{\alpha p} \left[ 4 (q \alpha \mid s r) - (qs \mid \alpha r) - (qr \mid s \alpha) \right] \\
(p,r) \in (\text{occ,act}) & : & \widetilde{w}_{pqrs} = 2 \sum\limits_{\beta \in \text{act}} \Gamma^{1,\text{act}}_{r \beta} \left[ 4 (q p \mid s \beta) -  (qs \mid p \beta) - (q \beta \mid sp) \right]
\end{eqnarray}

\hrule
\hrule

\end{document}